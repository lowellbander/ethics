\documentclass[]{article}

\usepackage[margin=1in]{geometry}
\usepackage{setspace}

\begin{document}

\title{Autobiography \& Analysis}
\author{Lowell Bander}
\maketitle

\onehalfspacing
\section{Personal Information}
My name is Lowell Bander. I am majoring in Computer Science, and am in the graduating class of 2016. I like to lift heavy things, consume psychedelic substances, listen to rap and indie folk, and have stimulating conversations. I would like to ultimately use technologies which interest me, such as virtual reality, augmented reality, machine learning, and software on the whole, to dismantle oppressive power structures, such as discrimination based on wealth, sexuality, gender identity, and race. I would similarly like to dismantle [America's] oligarchy, and make a positive impact on education by taking advantage of the technologies of the modern age.

\section{Ethical Background}
I was born to reform Jewish parents and raised in West Los Angeles. This no doubt has had an influence on the trajectory of my ethical values. For instance, having grown up in West Los Angeles, I have been exposed to individuals of various ethnic, racial, and religious backgrounds, which is likely why it is easy for me to perceive those of varied backgrounds as being equally deserving of the joys and privileges of life. 

However, I can appreciate that an individual who was born in, say, a racially homogenous environment may find it difficult to perceive individuals of different skin tones as being largely similar to him. Because of this, I recognize the inanity of labeling racists in the middle of America as ``evil", because I can see that they are just as much the product of their environment as I am of mine. Not only does verbiage such as ``evil" demonstrate an ignorance of the factors which shape an individual's perspective on the world, but it leads this individual to simplistic and ineffectual solutions, such as perhaps a violent crusade against those of different paradigms.

Not only have I grown up in the relatively liberal and diverse region of West Los Angeles, but I have traveled to more than half of the states in America, as well as Canada, Mexico, Scotland, and Israel. This has similarly given me the knowledge that individuals from around the world are not so fundamentally different, and so consequently labeling any grouping of individuals is an act which is likely unproductive at best, malicious at worst.

\section{Ethical Analysis of Diversity at HSSEAS}
\subsection{Situation}
First, a disclaimer. As a Computer Science student, I most certainly have a biased view of diversity at HSSEAS, for when I conjure up images of classrooms and the like, they will for the most part be images of CS classrooms, with CS students and CS professors. I am accordingly a poor candidate for assessing the diversity of HSSEAS on the whole.

That being said, I would describe the diversity climate within HSSEAS as being neutral in the explicit sense, in that it does not seem as though the school has made any efforts to improve diversity, nor has it committed any particularly egregious errors in intentionally prohibiting diversity. However, it is blatantly obvious that students are male by a vast majority, and that at least within the CS community, they are by and large either white, Asian, or Indian. Moreover, every professor I have ever had in this department has been male, and as far as I can remember, so has every chair of the department.

\subsection{Issues}
Admittedly, I got a little carried away in the preceding section and proceeded to outline what I find to be the most glaringly obvious issue with diversity within HSSEAS. The students within HSSEAS, or at least within the CS department, would appear to be male 9 times out of 10, and my professors have always been male. This most certainly has an effect on the female students in the school, as well as female prospectives who may visit our school while they are applying to colleges. Without role models in their field of interest who resemble them, it requires courage and perhaps a lack of risk aversion for a female student to take the steps necessary to join the school, and to stay within its ranks until graduation.

A near identical analysis can be applied to individuals whom belong to a race which is poorly represented, if at all, by the student body and the professors of the school. For example, an individual of Mexican or African descent may see few if any students or professors of their racial background, and are therefore exceedingly likely to be discouraged from joining the ranks of a school which seems as though it is ``not for them".

\subsection{Recommendations}
The school should establish a diverse task force with the intent of improving diversity within the school. For example, first they should list barriers to diversity, anything from explicit discrimination to lack of representation. The task force should then go on to ask students of varied backgrounds within the school and perhaps local high schools what might stop them from enrolling in UCLA HSSEAS. To top it off, the task force should perform further research on the matter in the form of recent articles and texts on the issue of diversity in higher education, perhaps particularly within education.

The task force should then initiate programs in K-12 programs locally, and perhaps throughout the country, with the goal of demonstrating to underrepresented demographics that engineering is well within their realm of opportunity. Lastly, the task force should regularly checkin with these individuals, both those within K-12 programs and those within HSSEAS to monitor how their attitudes towards the issue has changed over time as a result of the efforts of the task force. 

\end{document}